\section{Usage}
\label{sec:usage}
In this section, it will be briefly described how the module, when integrated, has to be used in order for it to work correctly.

At startup, the module shall be reset. Note that the reset signal is negedge triggered. The counter block \lstinline{cb} and the \lstinline{dout_valid} signals are set to zero.
Beware that the internal counter block is the key used by the module, therefore it is strongly recommended to set the key before encryption.

To set the key, the \lstinline{key} signal must contain the key and must be stable, and the \lstinline{key_in} signal must be asserted. Other signals are ignored. At least a clock cycle must be waited to allow for the key to be set correctly.

To encrypt (decrypt), the plaintext (ciphertext) must be provided byte per byte. In particular, the byte to encrypt (decrypt) must be fed through the \lstinline{din} signal, and the \lstinline{din_valid} signal must be asserted. The encryption (decryption) takes a single clock cycle per byte. The encryption (decryption) output \lstinline{dout} is to be considered valid only when \lstinline{dout_valid} is asserted. Each clock cycle an encryption (decryption) can be carried out.