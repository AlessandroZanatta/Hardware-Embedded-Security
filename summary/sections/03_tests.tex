\section{Testbench}
\label{sec:tests}
We will now describe the tests that were conducted on the designed hardware module.

\begin{enumerate}
    \item Given that the key space is small enough\footnote{Actually, the key space is definitely too small to provide any security.}, we are able to completely test the encryption (or, symmetrically, the decryption) of our module. Hence, a test performs the encryption of every possible character ($2^8$ possible characters) with every possible key ($2^8$ possible values).
    \item We have repeated the first test, but adding a random number (0 to 5) of idle cycles between the encryption of two values. This test was carried out to check that the signals were properly handled by our module.
    \item Finally, we have checked that the \cref{thm:enc_inv_dec} actually holds. To do so, we have encrypted a file ($\sim$39000 characters), and subsequently we have checked that its decryption (using the same key) was the same as the original file.
\end{enumerate}